\documentclass[12pt]{article}
\usepackage[utf8]{inputenc}
\usepackage{polski}
\usepackage{latexsym}
\usepackage[top=0.5in]{geometry}
\usepackage{enumerate}
\usepackage{ gensymb }

\date{}
\title{Projekt}
\author{Agnieszka Pawicka}
\frenchspacing
\begin{document}
\thispagestyle{empty}
\maketitle
\section{Jak używać}
\\\indent W archiwum znajduje się dołączony plik run.sh. Aby otworzyć program jako użytkownik init należy wpisać komendę ,,./run.sh  - -init'' . Następnie aby połączyć się z bazą: \\ \{ "open": \{ "database": "student", "login": "init", "password": "qwerty" \}\}. Później poprawne są jedynie wywołania funkcji leader, np.: \\ \{ "leader": \{ "timestamp": 1557473000, "password": "abc", "member": 1\}\}.
\\\indent Aby otworzyć program jako użytkownik app należy wpisać komendę \\,,./run.sh  - -app ''. Następnie aby połączyć się z bazą: \{ "open": \{ "database": "student", "login": "app", "password": "qwerty"\}\}. Od teraz poprawne są wywołania definiowane w poleceniu projektu, np.\\ \{ "support": \{ "timestamp": 1557475701, "password": "123", "member": 3, "action":600, "project":5000\}\}\\
\{ "trolls": \{ "timestamp": 1557477055 \}\}

\section{Opis}
\\Baza składa się z pięciu tabel:
\begin{enumerate}
\item members(id, cryptpwd, leader, lastactivity, upvotes, downvotes), gdzie:
\begin{itemize}
\item id - identyfikator użytkownika
\item cryptpwd - hasło zakodowane przy pomocy funkcji crypt z modułu pgcrypto (md5)
\item leader - true lub false
\item lastactivity - timestamp
\item upvotes, downvotes - wartości opisane w funkji trolls
\end{itemize}
\item actions(id, memberid, projectid, action, time)  - action przyjmuje wartości 's' (support), 'p' (protest)
\item votes(memberid, actionid, vote, time) - vote przyjmuje wartości 'u' (upvote), 'd' (downvote)
\item project(id, authority, creationtime)
\item ID - zawierające wszystkie identyfikatory użytkowników, akcji, projektów, organów władzy
\end{enumerate}

\\\\ Użytkownik init jest na prawach SUPERUSER, użytkownik app może wykonywać działania typu SELECT, INSERT, UPDATE.
\\Implementacja poszczególnych funkcji API:
\begin{itemize}
\item support/protest  $ timestamp$ $ member$ $ password$ $ action $ $project $ $[authority] $: \begin{enumerate}
\item sprawdzenie, czy  $member $  jest istniejącym ID członka, jeśli nie, dodawana jest nowa krotka do tabeli Member
\item sprawdzenie, czy hasło jest poprawne i czy członek jest aktywny, jeśli nie: zwrócenie błędu
\item sprawdzenie, czy podany projekt pojawił się wcześniej, jeśli nie pojawił się i nie podano authority lub pojawił się i podano inne authority niż poprzednio zwracany jest błąd
\item jeśli projekt nie pojawił się wcześniej dodawana jest krotka (project, authority) do tabeli Projects
\item dodawana jest nowa krotka do tabeli Actions o odpowiednich wartościach
\end{enumerate}
\item upvote/downvote  $timestamp $   $member $   $password $   $action $  :
\begin{enumerate}
\item sprawdzenie, czy  $member $  jest istniejącym ID członka, jeśli nie, dodawana jest nowa krotka do tabeli Member
\item sprawdzenie, czy hasło jest poprawne i czy członek jest aktywny, jeśli nie: zwrócenie błędu
\item sprawdzenie, czy istnieje akcja o podanym id, jeśli nie: zwracany jest błąd
\item dodawana jest nowa krotka do tabeli Votes (jeśli jest to drugie głosowanie danego członka w tej samej akcji, to nowa krotka złamie więzy klucza głównego tabeli Votes i nie zostanie dodana)
\end{enumerate}
\item actions  $timestamp $   $member $   $password $  [  $type $  ] [  $project $  [ $authority $  ] \\
votes  $timestamp $   $member $   $password $  [  $action $  ,  $project $  ]\\
projects  $timestamp $   $member $   $password $  [ authority ]: 
\begin{enumerate}
\item sprawdzenie, czy  $member $  jest istniejącym ID członka będącego liderem, jeśli nie, zwracany jest błąd,
\item sprawdzenie, czy hasło jest poprawne i czy członek jest aktywny, jeśli nie: zwrócenie błędu,
\item wywoływana jest odpowiednia funkcja SELECT
\end{enumerate}
\item trolls  $timestamp $ :
Wywoływana jest odpowiednia funkcja SELECT operująca na tabeli members. 
\end{itemize}\\
\\Dodatkowe funkcje ,,postgresowe'':
\begin{itemize}
\item wyzwalacze id\_trigger(), id\_trigger\_authority() wstawiające nowe krotki do tabeli id
\item wyzwalacz count\_votes() zliczający głosowania na akcje danego członka partii
\item funkcja active() sprawdzająca, czy członek partii wciąż jest aktywny
\item funkcja action\_type() konwertująca zawartość kolumny actions.action na wartości ,,protest'', ,,support''
\item funkcja timestamp\_cast() konwertująca unix timestamp 
\end{itemize}
\\Pomocnicze metody w pythonie:
\begin{itemize}
\item open\_connection(...) - łączy się z bazą
\item close\_connection(...) - zamyka połączenie z bazą
\item is\_leader(...) - sprawdza, czy dany użytkownik jest liderem
\item is\_member(...) - sprawdza, czy w bazie jest podany użytkownik, jeśli nie ma, próbuje dodać takiego
\item is\_active(...) - sprawdza, czy podany użytkownik jest aktywny
\item is\_action(...) - sprawdza, czy podana akcja jest w bazie
\item is\_projet(...) - sprawdza, czy podany projekt jest w bazie, jeśli nie, próbuje dodać
\item create\_leader(...) - dodaje nowego lidera
\item new\_action(...) - dodaje nową akcję
\item new\_vote(...) - dodaje nowy głos
\item trolls(...), votes(...), actions(...), projects(...) - wywołują odpowiednie SELECTy
\item taken\_args\_str(...), taken\_args(...) - pomocnicze funkcje do wykrywania opcjonalnych argumentów
\end{itemize}

\end{document}

